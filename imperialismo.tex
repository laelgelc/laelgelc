\begin{frame}{A manipulação linguística do Imperialismo}
    \begin{figure} [htpb]
        \centering
        \tikzstyle{block} = [rectangle, rounded corners, text=black, text centered, draw=black, font=\footnotesize, minimum height=.21in, minimum width=.65in]
        \tikzstyle{line} = [-latex,draw=black,line width=.4]
        \tikzstyle{line2} = [latex-latex,draw=black,line width=.4]
        \tikzstyle{line3} = [-latex,draw=black,line width=.4,dotted]
        \tikzstyle{line4} = [latex-latex,draw=black,line width=.4,dotted]
        \begin{tikzpicture}
            \node[block,fill=green!50,text=black, minimum width=.7in] (impe) at (10,11) {Imperialismo};


            \node[block,fill=green!50,text=black, minimum width=.7in] (geno) at (9,5) {
                \begin{tikzpicture}
                    \matrix [matrix of nodes, nodes=draw, minimum size=0.21in] {
                        Corporações & ONGs \\
                    };
                \end{tikzpicture}
            };



            \node[block,fill=green!50,text=black, minimum width=.7in] (geno) at (10,6) {
                \begin{tikzpicture}
                    \matrix [matrix of nodes, nodes=draw, minimum size=0.21in] {
                        Genocídio \\
                        Negacionismo \\
                        Anti-vacina \\
                        STF \\
                    };
                \end{tikzpicture}
            };

%            \node[block,fill=yellow!50,text=black, minimum width=.7in] (geno) at (10,6) {GenOutro};

            \node[block,fill=yellow!50,text=black, minimum width=.7in] (poli) at (5.5,4) {Político};
            \node[block,fill=yellow!50,text=black, minimum width=.7in] (mili) at (14.5,4) {Militar};
%            \node[block,fill=yellow!50,text=black, minimum width=.7in] (ling) at (10,5) {Linguístico};
        \end{tikzpicture}
%        \caption{A manipulação linguística do Imperialismo}
        \label{fig:imperialism}
    \end{figure}
\end{frame}