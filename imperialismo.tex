\begin{frame}{A manipulação linguística do Imperialismo}
    \begin{figure} [htpb]
        \centering
        \tikzstyle{block} = [rectangle, rounded corners, text=black, text centered, draw=black, font=\footnotesize, minimum height=.21in, minimum width=.65in]
        \tikzstyle{block1} = [rectangle, rounded corners, text=black, text centered, draw=none, font=\footnotesize, minimum height=.21in, minimum width=.65in]
        \tikzstyle{line} = [-latex,draw=black,line width=.4]
        \tikzstyle{line1} = [latex-latex,draw=black,line width=.4]
        \tikzstyle{line2} = [-latex,draw=black,line width=.4,dotted]
        \tikzstyle{line3} = [latex-latex,draw=black,line width=.4,dotted]
        \begin{tikzpicture}
            \node[block,fill=red!50,text=black, minimum width=.7in] (impe) at (10,10.25) {Imperialismo};
            \node[block,fill=orange!50,text=black, minimum width=.7in] (instituições) at (6.25,10) {
                \begin{tikzpicture}
                    \matrix [matrix of nodes, nodes={draw=none, minimum height=0.21in}] {
                        Corporações \\
                        ONGs \\
                    };
                \end{tikzpicture}
            };
            \node[block,fill=yellow!50,text=black, minimum width=.7in] (fina) at (6,8) {Financiamento};
            \node[block,fill=orange!50,text=black, minimum width=.7in] (manifestações) at (10,8) {
                \begin{tikzpicture}
                    \matrix [matrix of nodes, nodes={draw=none, minimum height=0.21in}] {
                        Genocídio \\
                        Negacionismo \\
                        Anti-vacina \\
                        STF \\
                    };
                \end{tikzpicture}
            };
            \node[block,fill=orange!50,text=black, minimum width=.7in] (abst) at (12.75,8) {Abstrato?};
            \node[block,fill=orange!50,text=black, minimum width=.7in] (base) at (7,6.5) {Base Concreta};
            \node[block,fill=red!50,text=black, minimum width=.7in] (econ) at (7,5) {Econômico};
            \node[block,fill=yellow!50,text=black, minimum width=.7in] (poli) at (5.5,3.25) {Político};
            \node[block,fill=yellow!50,text=black, minimum width=.7in] (mili) at (8.5,3.25) {Militar};
            \node[block,fill=green!50,text=black, minimum width=.7in] (ling) at (10,5.25) {Linguístico};
            \node[block1,text=orange, minimum width=.7in] (anes) at (13,6) {Anestesia};
            \node[block1,text=orange, minimum width=.7in] (apoi) at (10,4.25) {Apoio Popular};

            \draw[decorate,decoration={brace,amplitude=10pt}] (manifestações.north east) -- (manifestações.south east);
            \draw[decorate,decoration={brace,amplitude=10pt,mirror}] (ling.south west) -- (ling.south east);
            \draw[line] (instituições) to [in=180,out=0] (impe);
            \draw[line] (instituições) to [in=90,out=270] (fina);
            \draw[line] (fina) to [in=180,out=225] (econ);
            \draw[line] (base) to [in=90,out=270] (econ);
            \draw[line] (poli) to [in=225,out=90] (econ);
            \draw[line] (mili) to [in=315,out=90] (econ);
            \draw[line] (ling) to [in=0,out=180] (econ);
            \draw[line1] (ling) to [in=270,out=45] (abst);
            \draw[line] (ling) to [in=180,out=30] (anes);
        \end{tikzpicture}
%        \caption{A manipulação linguística do Imperialismo}
        \label{fig:imperialismo}
    \end{figure}
\end{frame}